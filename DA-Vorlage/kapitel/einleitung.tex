% einleitung.tex
\chapter{Einleitung}
\section{Motivation und Hintergrund}
Maschinelles Lernen und Vorhersagen werden immer mehr in unser Leben integriert. Hierbei entsteht zum einen der Anspruch an variable, nicht statische Systeme, zum anderen die Notwendigkeit kompakter und energieeffizienter Lösungen.\\
Aufgrund der immer weiter wachsenden Datenmengen stoßen herkömmliche Central Processing Units (CPUs) mittlerweile an Ihre Grenzen, denn durch materialbedingte Limitierung kann ihre Rechenkapazität so gut wie nicht mehr erhöht werden. Daher geht man dazu über, Mehrkernprozessoren zu entwickeln, die ihre Geschwindigkeit über parallele Threads erreichen. Diese haben jedoch einen vergleichsweise hohen Energieverbrauch.\\
Field Programmable Gate Arrays (FPGAs) bieten in diesem Zusammenhang einen guten Kompromiss zwischen Flexibilität in der Programmierbarkeit und Energieeffizienz. Der Vorteil der FPGAs zeigt sich in der deutlich höheren Parallelität gegenüber CPUs, sodass trotz der geringeren Taktfrequenz eine große Menge an Daten schnell verarbeitet werden kann.\\
Die logistische Regression ist für die Optimierung auf FPGAs in dem Sinne gut geeignet, da sie eine einfache Art von neuronalem Netz darstellt und somit gut in der FPGA-Logik darstellbar ist. Sie weist zum Beispiel durch Datenparallelität bzw. Parallelisierung von Batches, Feature- oder Hyperparameter-Berechnung eine hohe Parallelisierbarkeit auf.\\\\
Moderne FPGAs können über die PCIe-Schnittstelle als CO-Prozessor in ein System eingebunden werden, sodass deren Parallelität und Energieeffizient ausgenutzt werden können. Dank des hohen Durchsatzes der Schnittstelle muss hierbei nicht auf eine komplexe variable Vorbereitung der Daten durch die CPU im laufenden Betrieb verzichtet werden.

\section{Aufbau der Arbeit}